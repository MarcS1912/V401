\section{Diskussion}
Die experimentell bestimmte Wellenlänge ergab sich zu $\lambda_\text{exp}=\qty{821(2)}{\nm}$.
Es ist allerdings bekannt, dass der Laser nahezu monochromatisches Licht mit 
einer Wellenlänge von $\lambda_\text{theo}=\qty{680}{\nm}$ emittiert.
Die relative Abweichung nach Gleichung \eqref{eq:delta} ist demnach mit 20.8\% relativ hoch
und der theoretische Wert liegt auch weit außerhalb des 1-$\sigma$ Intervalls.
Theoretisch hätten also deutlich mehr Minima detektiert werden müssen, als gemessen worden sind.
Dementsprechend liegen systematische Fehler vor, die die gesamte Messergebnisse nach unten verschoben haben.
Sehr Auffällig ist, dass der erste gemessenen Wert tatsächlich noch deutlich größer als die anderen gemessenen Werte sind.
Dementsprechend könnte zwischen der ersten und zweiten Messung die Messvorrichtung ungewollt 
verändert worden sein. Die Justierung der Spiegel, die so eingestellt sein sollte, dass ein Interferenzmuster mit klar zu unterscheidbaren Maxima 
zu erkennen sind, die für den Detektor auch zusätzlich richtig orientiert sind, ist im Allgemeinen sehr störanfällig.
Da bereits kleinste Änderungen an den Justierrädern einen klar erkennbaren Einfluss auf das Interferenzmuster haben.
Dementsprechend könnten bereits unbewusste Berührungen oder Streifungen mit dem der Messvorrichtungen die Justierung 
gestört haben, wodurch nicht jedes Minimum richtig detektiert werden konnte.
Eine weitere Ursache könnte darin bestehen, dass der Motor der Längenverstellung zu schnell lief.
Da das Zählwerk Strompulse misst, müssen diese möglichst gut unterscheidbar sein. Wenn zu schnell Pulse
auftreten, kann das Zählwerk diese möglicherweise nicht mehr gut erfassen.\\
\\
\noindent Der Brechungsindex ergab sich je nach Druckänderungsrichtung und insgesamt zu 
\begin{align*}
    n_1&=\qty{1.000223(0.000014)}{}\\
    n_2&=\qty{1.000290(0.000012)}{}\\
    n&=\qty{1.000261(0.000009)}{}.\\
\end{align*}
Bei Normalbedingung, also bei 0°C und dem Atmosphärendruck auf Meereshöhe,
sollte der Brechungsindex bei $n_\text{theo}=\qty{1.0002911}{}$ liegen \cite{Brechungsindex}.
Die direkte relative Abweichung über Formel \eqref{eq:delta} ergibt sehr geringe Abweichungen von
\begin{align*}
    \Delta_1&=0,0068\%\\
    \Delta_2&=0,0001\%\\
    \Delta&=0,0031\%.\\
\end{align*}
Diese geringe Abweichung ist an sich aber kein guter Indikator, ob die Messung gut oder schlecht war. Sie sagt nur aus, dass 
alle experimentellen Messwerte in einer ähnlichen Größenordnung wie der theoretische Wert vom Brechungsindex im Vakuum abweichen.
Das an sich ist schonmal nicht schlecht, reicht für eine Beurteilung aber noch nicht aus. Ein besserer Indikator stellt die 
experimentell bestimmte Abweichung gegen die theoretische Abweichung $n'=n-1$ zum Vakuum dar.
Damit ergeben sich als relative Abweichungen
\begin{align*}
    \Delta'_1&=23,5\%\\
    \Delta'_2&=0,3\%\\
    \Delta'&=10,5\%.\\
\end{align*}
Hieran lässt sich schon deutlich besser ablesen, dass nur der Brechungsindex, der beim hereinlassen der Luft ermittelt wurde,
ein wirklich gutes Ergebnis liefert. Es ist auch die einzige Messung, bei der der theoretische 
Wert im 1-$\sigma$-Konfidenzintervall der Messung liegt.
Dieses Ergebnis sollte allerdings auch nicht übermäßig stark bewertet werden. Es ist zu beachten, dass 
nach der ersten Messung zur Bestimmung der Wellenlänge keine erneute Justierung vorgenommen worden ist. 
Das heißt, dass ein systematischer Fehler aus der ersten Messreihe auch hier präsent sein könnte und sich mit anderen systematischen Fehlern aufhebt. 
In dem Fall wäre das gute Ergebnis für den Brechungsindex eher Zufall.
Unter der Annahme, dass dieser Fall nicht eintrifft, kann sich jetzt noch darüber Gedanken gemacht werden, warum bei der Evakuierung soviel schlechtere 
Ergebnisse produziert werden, als bei der Öffnung des Ventils.
Der hauptsächliche Unterschied besteht nun hauptsächlich in der Stetigkeit des Luft Zu-/Abflusses.
Bei der Benutzung der Vakuumpumpe wird in bei jedem Pumpvorgang der Messkammer sehr abrupt viel 
Luft entzogen, während bei Öffnung des Ventils die Luft gleichmäßig in die Kammer fließen kann.
Das könnte nun wieder ein Problem für das Zählwerk darstellen.
Bei der abrupten Evakuierung könnten manche sehr kurz hintereinander auftretenden Pulse als einer detektiert werden,
da der Strom noch nicht weit genug unter einen Schwellwert gefallen sein könnte.
