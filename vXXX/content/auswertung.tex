\section{Auswertung}
\label{sec:Auswertung}
\subsection{Fehlerrechnung}
Der Mittelwert einer physikalischen Größe $a$ mit $N$ Einzelmesswerten ist gegeben durch
\begin{equation}\label{eq:mean}
    \overline{a}=\frac{1}{N}\sum_{i=1}^Na_\text{i}\,.
\end{equation}
Die Standardabweichung eines Mittelwerts zu einer physikalischen Größe a bestimmt sich mit
\begin{equation}\label{eq:std}
    \Delta{a}=\sqrt{\frac{1}{N(N-1)}\sum_{i=1}^N\left(a_\text{i}-\overline{x}\right)}\,.
\end{equation}
Für fehlerbehaftete und voneinander unabhängige Eingangsgrößen $x_\text{k}$ mit {${k=1\text{,}\,\ldots\text{,}\,n}$} und den Ungenauigkeiten $\Delta x_\text{k}$ lautet die Ungenauigkeit der abgeleiteten Größe $f$ nach der gaußschen Fehlerfortpflanzung
\begin{equation}\label{eq:gauss}
    \Delta f=\sqrt{\sum_{k=1}^{n}\biggl(\frac{\partial f}{\partial x_\text{k}}\biggr)^2(\Delta x_\text{k})^2}\,.
\end{equation}
Der gewichtete Mittelwert einer fehlerbehafteten Größe $a\pm b$ mit 2 Einzelmesswerten ist gegeben durch
\begin{equation}\label{eq:gmean}
    \overline{a}=\frac{p_1\cdot a_1+p_2\cdot a_2}{p_1+p_2},
\end{equation}
wobei $p_1=1/b_1^2$ und $p_2=1/b_2^2$ die Gewichte sind.
Der Fehler des Mittelwerts bestimmt sich dann durch 
\begin{equation}\label{eq:gstd}
    \Delta a= \frac{1}{\sqrt{p1+p2}}.
\end{equation}
Die relative Abweichung zwischen einem experimentell bestimmten Wert $a_\text{exp}$ und der theoretischen Vorhersage $a_\text{theo}$ wird 
durch 
\begin{equation}\label{eq:delta}
    \Delta = \frac{|a_\text{exp}-a_\text{theo}|}{a_\text{theo}}
\end{equation}
ermittelt.
\subsection{Bestimmung der Wellenlänge eines Lasers}
Das Interferometer soll zunächst dafür verwendet werden, die Wellenlänge des verwendeten Lasers zu bestimmen.
Dafür wird 10 Mal die Anzahl auftretender Interferenzminima bei einer festen Änderung des Lichtwegs um eine entsprechende Veränderung
an einer Mikrometerschraube um $d=\qty{5}{\mm}$ gemessen. Dies ist allerdings nicht immer gelungen, weswegen zunächst zwei Teilmessreihen 
mit je einmal $d_1=$5,1mm und $d_2=$5mm betrachtet werden. Die entsprechenden gemessenen Minima sind in Tabelle \ref{tab:Laenge} eingetragen.
\begin{table}[H]
  \centering
  \caption{Messwerte zur Bestimmung der Wellenlänge eines Lasers.}
  \label{tab:Laenge}
  \begin{tabular}{S[table-format=1.1] S[table-format=4]}
    \toprule
    {$d$/mm }& {Interferenzminima $z$} \\
    \midrule
    5,1 & 2776 \\
    5,1 & 2487 \\
    5 & 2438 \\
    5 & 2427 \\
    5 & 2436 \\
    5 & 2409 \\
    5 & 2435 \\
    5 & 2418 \\
    5 & 2430 \\
    5 & 2413 \\
    \bottomrule
  \end{tabular}
\end{table}
\noindent Zunächst werden gemäß Gleichung \eqref{eq:mean} und \eqref{eq:std} der Mittelwert und die Standardabweichung des 
Mittelwerts der gemessenen Minima für beide Mikrometerschrauben-Verstellungen berechnet. Es ergeben sich 
\begin{align*}
    z_1&=\qty{2567(105)}{}\\
    z_2&=\qty{2426(4)}{}.\\
\end{align*}
Nun muss aus der Verstellung der Mikrometerschraube noch die zusätzliche Weglänge des Lichts ermittelt werden. Dabei muss die Übersetzung
$\text{ü}=\num{5.017}$ und die Tatsache, dass das Licht jeden zusätzlichen Weg zweimal zurücklegen muss, beachtet werden.
Damit ergibt sich der Lichtweg zu 
\begin{equation}
  d_\text{Licht}=\frac{2\cdot d}{\text{ü}}.
\end{equation}
Der Fehler berechnet sich über die gleiche Formel, wobei $\Delta d=\qty{0.01}{\mm}$ einem Skalenanteil entspricht.
Das bedeutet für die beiden Verstellungen der Mikrometerschraube ein jeweiliger Lichtweg von 
\begin{align*}
  d_\text{Licht,1}&=\qty{2.020(0.004)}{\mm}\\
  d_\text{Licht,2}&=\qty{1.993(0.004)}{\mm}.\\
\end{align*}
Aus Formel \eqref{eq:} über den Zusammenhang auftretender destruktive Interferenzen und dem Verhältnis vom Lichtweg zur Wellenlänge lässt sich mit dem Mittelwert der gemessenen Minima und der 
gemessenen Wegstrecke des Lichts die Wellenlänge $\lambda$ bestimmen. Der Fehler der Wellenlänge ergibt sich gemäß Gaußscher Fehlerfortpflanzung \eqref{eq:gauss} zu 
\begin{equation*}
    \Delta \lambda =\overline{\lambda}\sqrt{\left(\frac{\Delta d_\text{Licht}}{\overline{d_\text{Licht}}}\right)^2+\left(\frac{2\Delta z }{2\overline{z}+1}\right)^2}
\end{equation*}
Für die einzelnen Wegunterschiede ergeben sich dann für die Wellenlänge des Lasers 
\begin{align*}
  \lambda_1&=\qty{787(32)}{\nm}\\
  \lambda_2&=\qty{822(2)}{\nm}.\\
\end{align*}
Nun kann das gewichtete Mittel über beide Ergebnisse gemäß den Gleichungen \eqref{eq:gmean} und \eqref{eq:gstd}
gebildet werden. Dann ergibt sich die aus der Stichprobe resultierende Wellenlänge zu 
\begin{equation*}
    \lambda=\qty{821(2)}{\nm}.
\end{equation*}
\subsection{Bestimmung des Brechungsindex in Luft}
Für die Bestimmung des Brechungsindex von Luft werden bei einem Druckunterschied von $\Delta p=p-p'=500$mmHg$=666.5$mbar die 
auftretenden Interferenzminima gezählt. Die entsprechenden Messwerte sind in Tabelle \ref{tab:Index} abgebildet, wobei die ungeraden 
Einträge den Evakuierungsprozess beschreiben, während bei den geraden Einträgen das Ventil geöffnet worden
ist um die Luft wieder hereinzulassen. Da bereits ohne Auswertung für beide Fälle 
ein sichtbar unterschiedliches Verhalten zu erkennen ist, werden die Fälle auch zunächst unabhängig voneinander betrachtet.
Der Evakuierungsfall wird mit dem Index 1 und der Einlass der Luft mit dem Index 2 bezeichnet.

\begin{table}[H]
  \centering
  \caption{Messwerte zur Bestimmung des Brechungsindex von Luft.}
  \label{tab:Index}
  \begin{tabular}{S[table-format=4]}
    \toprule
    {Interferenzminima $z$ }\\
    \midrule
    23 \\
    25 \\
    19 \\
    25 \\
    18 \\
    26 \\
    18 \\
    26 \\
    21 \\
    27 \\
    \bottomrule
  \end{tabular}
\end{table}

\noindent Nun muss für beide Fälle wieder eine Mittelwertbildung gemäß den Gleichungen 
\eqref{eq:mean} und \eqref{eq:std} durchgeführt werden. Es ergibt sich für $z$
\begin{align*}
  z_1&=\qty{19.8(1.0)}{}\\
  z_2&=\qty{25.8(0.4)}{}.\\
\end{align*}
Nun soll über Gleichung \eqref{eq:} der gemessene Unterschied des Brechungsindex bestimmt werden. 
Dafür wird der theoretische Wert der Wellenlänge des Lasers von $\lambda=\qty{680}{\nm}$
und die Dicke der Messzelle $b=\qty{50}{\mm}$ benutzt. Im folgenden werden Fehler mit einem $D$ angegeben, um 
Verwirrungen zu vermeiden. Der Fehler der Wellenlängenänderung $D\Delta n$ ergibt sich über die gleiche Formel, 
wobei statt $z$ $Dz$ eingesetzt wird.
Damit ergibt sich für die Änderung des Brechungsindex bei Änderung des Drucks
\begin{align*}
  \Delta n_1&=\qty{1.34(0.07)e-4}{}\\
  \Delta n_2&=\qty{1.75(0.03)e-4}{}.\\
\end{align*}
Nun soll aus der Änderung des Brechungsindex bei Änderung des Luftdrucks der Brechungsindex 
von Luft bei Normalbedingung $T_0=\qty{273.15}{\kelvin}$ und $p_0=\qty{1013.2}{}\text{mbar}$ bestimmt werden.
Die Temperatur während der Messung ist vergessen worden zu messen. Daher muss sich hier auf eine grobe Schätzung von 
$T=297.15K$ beziehungsweise 24°C verlassen werden. Es wird vermutet, dass sich die Temperatur in einem 2 K Intervall um diesen Wert aufhält.
Daher wird ein Fehler von $DT=\qty{2}{\kelvin}$ angegeben.
Nun lässt sich der Brechungsindex über die Formel 
\begin{equation*}
  n=1+\Delta n \frac{T\cdot p_0}{T_0\cdot \Delta p}
\end{equation*}
bestimmen. Der Fehler pflanzt sich gemäß der Gaußschen Fehlerfortpflanzung \eqref{eq:gauss} zu 
\begin{equation}
    Dn= (n-1)\sqrt{\left(\frac{D\Delta n}{\Delta n}\right)^2+\left(\frac{D\Delta p}{\Delta p}\right)^2+\left(\frac{DT}{T}\right)^2}
\end{equation}
fort, wobei $D\Delta p=20\text{mmHg}=26.66\text{mbar}$ ist und 2 Skalenanteilen auf dem Manometer entspricht. Der Fehler 
resultiert aus dem ungenauen Ablesen von Start und Zieldruck. 
Damit ergeben sich aus den Messungen bei Evakuierung und hereinlassen der Luft die 
Brechungsindizes 
\begin{align*}
  n_1&=\qty{1.000223(0.000014)}{}\\
  n_2&=\qty{1.000290(0.000012)}{}.\\
\end{align*}
Nun kann noch das gewichtete Mittel beider Messungen nach Gleichung \eqref{eq:gmean} und \eqref{eq:gstd} gebildet werden. Damit ergibt sich als 
Brechungsindex für alle 10 Messungen
\begin{equation*}
    n=\qty{1.000261(0.000009)}{}
\end{equation*}
