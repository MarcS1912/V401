\section{Auswertung}
\label{sec:Auswertung}
\subsection{Fehlerrechnung}
Der Mittelwert einer physikalischen Größe $a$ mit $N$ Einzelmesswerten ist gegeben durch
\begin{equation}\label{eq:mean}
    \overline{a}=\frac{1}{N}\sum_{i=1}^Na_\text{i}\,.
\end{equation}
Die Standardabweichung eines Mittelwerts zu einer physikalischen Größe a bestimmt sich mit
\begin{equation}\label{eq:std}
    \Delta{a}=\sqrt{\frac{1}{N(N-1)}\sum_{i=1}^N\left(a_\text{i}-\overline{x}\right)}\,.
\end{equation}
Für fehlerbehaftete und voneinander unabhängige Eingangsgrößen $x_\text{k}$ mit {${k=1\text{,}\,\ldots\text{,}\,n}$} und den Ungenauigkeiten $\Delta x_\text{k}$ lautet die Ungenauigkeit der abgeleiteten Größe $f$ nach der gaußschen Fehlerfortpflanzung
\begin{equation}\label{eq:gauss}
    \Delta f=\sqrt{\sum_{k=1}^{n}\biggl(\frac{\partial f}{\partial x_\text{k}}\biggr)^2(\Delta x_\text{k})^2}\,.
\end{equation}
Der gewichtete Mittelwert einer fehlerbehafteten Größe $a\pm b$ mit 2 Einzelmesswerten ist gegeben durch
\begin{equation}\label{eq:gmean}
    \overline{a}=\frac{p_1\cdot a_1+p_2\cdot a_2}{p_1+p_2},
\end{equation}
wobei $p_1=1/b_1^2$ und $p_2=1/b_2^2$ die Gewichte sind.
Der Fehler des Mittelwerts bestimmt sich dann durch 
\begin{equation}\label{eq:gstd}
    \Delta a= \frac{1}{\sqrt{p1+p2}}.
\end{equation}
\subsection{Bestimmung der Wellenlänge eines Lasers}
Das Interferometer soll zunächst dafür verwendet werden, die Wellenlänge des verwendeten Lasers zu bestimmen.
Dafür wird 19 Mal die Anzahl auftretender Interferenzminima bei einer festen Änderung des Lichtwegs um eine entsprechende Veränderung
an einer Mikrometerschraube um $d=\qty{5}{\mm}$. Dies ist allerdings nicht immer gelungen, weswegen zunächst zwei Teilmessreihen 
mit je einmal $d_1=$5,1mm und $d_2=$5mm. Die entsprechenden gemessenen Minima sind in Tabelle \ref{tab:Laenge} eingetragen.
\begin{table}[H]
  \centering
  \caption{Messwerte zur Bestimmung der Wellenlänge eines Lasers.}
  \label{tab:Laenge}
  \begin{tabular}{S[table-format=1.1] S[table-format=4]}
    \toprule
    {$d$/mm }& {Interferenzminima $z$} \\
    \midrule
    5,1 & 2776 \\
    5,1 & 2487 \\
    5 & 2438 \\
    5 & 2427 \\
    5 & 2436 \\
    5 & 2409 \\
    5 & 2435 \\
    5 & 2418 \\
    5 & 2430 \\
    5 & 2413 \\
    \bottomrule
  \end{tabular}
\end{table}
\noindent Zunächst werden gemäß Gleichung \eqref{eq:mean} und \eqref{eq:std} der Mittelwert und die Standardabweichung des 
Mittelwerts der gemessenen Minima für beide $d$ berechnet. Es ergeben sich 
\begin{align*}
    z_1&=\qty{2567(105)}{}\\
    z_2&=\qty{2426(4)}{}.\\
\end{align*}
Nun muss aus der Verstellung der Mikrometerschraube noch die zusätzliche Weglänge des Lichts ermittelt werden. Dabei muss die Übersetzung
$\text{ü}=\num{5.017}$ und die Tatsache, dass das Licht jeden zusätzlichen Weg zweimal zurücklegen muss, beachtet werden.
Damit ergibt sich der Lichtweg zu 
\begin{equation}
  d_\text{Licht}=\frac{2\cdot d}{\text{ü}}.
\end{equation}
Der Fehler berechnet sich über die gleiche Formel, wobei $\Delta d=\qty{0.01}{\mm}$ einem Skalenanteil entspricht.
Das bedeutet für die beiden Verstellungen der Mikrometerschraube ein Lichtweg von 
\begin{align*}
  d_\text{Licht,1}&=\qty{2.020(0.004)}{\mm}\\
  d_\text{Licht,2}&=\qty{1.993(0.004)}{\mm}.\\
\end{align*}
Aus Formel \eqref{eq:} für destruktive Interferenzen lässt sich mit dem Mittelwert der gemessenen Minima und der 
gemessenen Wegstrecke des Lichts die Wellenlänge $\lambda$ bestimmen. Der Fehler der Wellenlänge ergibt sich gemäß Gaußscher Fehlerfortpflanzung \eqref{eq:gauss} zu 
\begin{equation*}
    \Delta \lambda =\overline{\lambda}\sqrt{\left(\frac{\Delta d_\text{Licht}}{\overline{d_\text{Licht}}}\right)^2+\left(\frac{2\Delta z }{2\overline{z}+1}\right)^2}
\end{equation*}
Für die einzelnen Wegunterschiede ergeben sich dann für die Wellenlänge des Lasers 
\begin{align*}
  \lambda_1&=\qty{787(32)}{\nm}\\
  \lambda_2&=\qty{822(2)}{\nm}.\\
\end{align*}
Nun kann das gewichtete Mittel über beide Ergebnisse gemäß den Gleichungen \eqref{eq:gmean} und \eqref{eq:gstd}
gebildet werden. Dann ergibt sich die resultierende Wellenlänge der Stichprobe zu 
\begin{equation*}
    \lambda=\qty{821(2)}{\nm}.
\end{equation*}
\subsection{Bestimmung des Brechungsindex in Luft}

\begin{table}[H]
  \centering
  \caption{Messwerte zur Bestimmung des Brechungsindex von Luft.}
  \label{tab:Index}
  \begin{tabular}{S[table-format=4]}
    \toprule
    {Interferenzminima $z$ }\\
    \midrule
    23 \\
    25 \\
    19 \\
    25 \\
    18 \\
    26 \\
    18 \\
    26 \\
    21 \\
    27 \\
    \bottomrule
  \end{tabular}
\end{table}